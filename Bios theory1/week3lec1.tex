\documentclass[10pt]{article}
\usepackage[utf8]{inputenc}
\usepackage[T1]{fontenc}
\usepackage{amsmath}
\usepackage{amsfonts}
\usepackage{amssymb}
\usepackage[version=4]{mhchem}
\usepackage{stmaryrd}
\usepackage{graphicx}
\usepackage[export]{adjustbox}
\usepackage{preamble}

\graphicspath{ {./images/} }

\begin{document}

$u_{1}, \ldots u_{n} \sim $ uniform $(0,1)$

$$
\begin{aligned}
& f_{u_{(k)}}(t)=\left(\begin{array}{c}
n \\
k-1,1, n-k
\end{array}\right) t^{k-1}(1-t)^{n-k}, 0 \leq t \leq 1 \\
& f_{u(i)}, u_{(k)}(s, t)=\left(\begin{array}{l}
n \\
i-1,1, k-i-1,n-k
\end{array}\right) s^{i-1}(t-s)^{k-i-1}(1-t)^{n-k}, 0 \leq s \leq t \leq 1
\end{aligned}
$$

(c) $f_{u(1) \ldots u(n)}\left(s_{1}, s_{2}, s_{n}\right)=\left(\begin{array}{l}
    n \\
    1,1 \ldots  ,1
    \end{array}\right) f\left(s_{1}\right) \cdots f\left(s_{n}\right)$

$$
=n ! \cdot 1 \cdot 1 \cdot 1=n ! \quad 0 \leqslant S_{1} \leq S_2 \quad S_{n} \leqslant 1
$$

\begin{theorem}
for any random sample $x_{1}, x_{n}$ of cont .r.v.S having $c d f F$ and density $f$, the order statistics $X_{(k)}, 
1 \leq k \leq n$, has density $.f_{x(k)}(t)=\left(\begin{array}{l}
    n \\
    k-1,1,n-k
    \end{array}\right) F(t)^{k-1} f(t)[1-F(t)]^{n-k}$ 
for all $t$. and the joint density of $\left(X_{(i)}, X_{(k)}\right)$ is

Finally, the joint density of $\left(X_{(1)}, X_{(2)} \cdots X_{(0)}\right)$
    
    $$
    f_{x_{(1)} \ldots x_{(n)}}\left(S_{1}, S_{2}, \cdots S_{n}\right)=n ! f\left(S_{1}\right) 
    \cdot f\left(S_{2}\right) \cdots f\left(S_{n}\right)=n ! \prod_{i=1}^{n} f\left(S_{i}\right), 
    S_{1}\leq S_{2}\leq \ldots S_{k}
    $$
    
\end{theorem}
Comment: the prob of any ties is zero, whenever $F$ is continuum dist. 
\begin{example}
    How would you find the list of the sample Range?
\end{example}

$$
x_{1}\ldots x_{n} \sim   F \quad R_{n}=x_{(n)}-x_{(1)}
$$

(1) $f_{X(1), X_{(n)}}$

(2) $\left(X_{(1)}, X_{(n)}\right) \rightarrow\left(X_{(1)}, R_{n}\right)$

(3) marginal $R_{n}$.

\begin{example}
    ( by stems reliability)

$Y_{1}, Y_{2} \ldots=$ life time of sample electric components (Suppisise indepenonont) 
$
\sim \operatorname{Exp}(\text { mean }=\theta)
$
\end{example} 


suppose we organize these in a more complicated system.

$T=$ life time of the system.

(a) Simple series circuit. w/ m component.

$$
\begin{aligned}
T & =\min \left(Y_{1}, Y_{m}\right) \\
F_{T}(t) & =P(T \leq t)=1-P(T>t)=1-\left(1-F_{y}(t)\right)^{m}=1-e^{-m t / \theta}, t >0
\end{aligned}
$$

(b) Simple parallel circus w/ m component.

$$
\begin{aligned}
T & =\max \left(Y_{1}, \ldots Y_{m}\right) \\
F_{T}(t) & =P(T \leq t)=F_{Y}(t)^{m}=\left(1-e^{-t / \theta}\right)^{m}, t>0 .
\end{aligned}
$$

(C) Parallel systems in series

m units are in series and each unit has $k$ components in parallel.
$ Y_{i j}=j^{\text {th }} \text { comport of } i^{\text {th }} \text { unit. }$
$ T=\min _{1 \leq i \leq m}\left(\max _{1 \leq j \leq n}\left(Y_{i j}\right)\right) 
$
$$
\begin{aligned}
F_{T}(t)&= 1-\left[1-F_{u(t)}(t)\right]^{m} \\
&=1-\left[1-\left(1-e^{-t / \theta}\right)^{k}\right]^{m}, t>0
\end{aligned}
$$


\end{document}